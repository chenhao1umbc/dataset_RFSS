\documentclass[twocolumn]{article}
\usepackage[margin=0.75in]{geometry}
\usepackage{amsmath,amssymb}
\usepackage{graphicx}
\usepackage{cite}
\usepackage{url}
\usepackage{array}

% Title and authors
\title{RFSS: A Comprehensive Multi-Standard RF Signal Source Separation Dataset with Advanced Channel Modeling}

\author{
Hao Chen, Rui Jin, and  Dayuan Tan
}

\date{\today}

\begin{document}

\maketitle

\begin{abstract}
The rapid evolution of wireless communication systems has created complex electromagnetic environments where multiple cellular standards (2G/3G/4G/5G) coexist, necessitating advanced signal source separation techniques. We present RFSS (RF Signal Source Separation), a comprehensive open-source dataset containing 52,847 realistic multi-standard RF signal samples with complete 3GPP standards compliance. Our framework generates authentic baseband signals for GSM, UMTS, LTE, and 5G NR with advanced channel modeling including multipath fading, MIMO processing up to 8×8 antennas, and realistic interference scenarios. Experimental validation demonstrates superior performance of CNN-LSTM architectures achieving 26.7 dB SINR improvement in source separation tasks, significantly outperforming traditional ICA (15.2 dB) and NMF (18.3 dB) approaches. The RFSS dataset enables reproducible research in RF source separation, cognitive radio, and machine learning applications while maintaining complete open-source accessibility.
\end{abstract}

\textbf{Keywords:} RF source separation, multi-standard signals, 3GPP compliance, MIMO, machine learning, open source dataset

\section{Introduction}

The proliferation of wireless communication technologies has created increasingly complex electromagnetic environments where multiple cellular standards operate simultaneously within overlapping frequency bands. Modern wireless ecosystems encompass legacy 2G GSM systems, 3G UMTS networks, 4G LTE infrastructure, and emerging 5G NR deployments, all coexisting in shared spectrum resources \cite{cabric2004implementation,mitola1999cognitive}. This heterogeneous landscape represents a fundamental paradigm shift from traditional single-standard communication systems to dynamic multi-standard coexistence scenarios that demand sophisticated signal processing and interference management techniques.

The complexity of modern RF environments is further amplified by the diverse modulation schemes, bandwidth allocations, and power control mechanisms employed across different cellular generations. GSM systems utilize constant-envelope GMSK modulation with 200 kHz channels, while UMTS employs wideband CDMA with 5 MHz carriers. LTE and 5G NR systems implement OFDMA with flexible bandwidth configurations ranging from 1.4 MHz to 100 MHz, creating a rich tapestry of spectral signatures that must coexist within limited spectrum resources. 

\begin{figure}[t]
\centering
\includegraphics[width=0.95\columnwidth]{../figures/stft_spectrograms.pdf}
\caption{STFT spectrograms of multi-standard RF signals showing time-frequency characteristics. Individual spectrograms demonstrate the unique spectral signatures of GSM, UMTS, LTE, and 5G NR signals, revealing their distinct modulation schemes and bandwidth utilization patterns essential for source separation research.}
\label{fig:signal_spectrograms}
\end{figure}

This heterogeneous landscape presents fundamental challenges for wireless system design, spectrum management, and interference mitigation \cite{haykin2005cognitive,goldsmith2009breaking}. The emergence of machine learning applications in wireless communications has intensified the demand for comprehensive, realistic RF datasets that accurately represent multi-standard coexistence scenarios \cite{oshea2017introduction,west2017deep}.

Current research in cognitive radio, spectrum sensing, and automated signal classification relies heavily on the availability of high-quality training data \cite{oshea2018over,oshea2016convolutional,rajendran2018deep}. However, existing RF datasets suffer from significant limitations that constrain their applicability to advanced research applications. The widely-used RadioML dataset focuses primarily on modulation recognition without comprehensive multi-standard coverage \cite{oshea2016radio}. Commercial solutions like MATLAB 5G toolbox provide single-standard generation but lack open-source accessibility \cite{west2017deep}. GNU Radio offers partial implementation with limited 3GPP compliance verification \cite{blossom2004gnu}.

Figure \ref{fig:signal_spectrograms} demonstrates the comprehensive time-frequency analysis of multi-standard signals generated by our RFSS framework, revealing the distinct spectral characteristics of 2G through 5G waveforms essential for advanced source separation research.

To address these research challenges and dataset limitations, we developed the RFSS framework with three primary design objectives: (1) comprehensive coverage of all major cellular standards with rigorous 3GPP compliance, (2) realistic multi-standard coexistence scenarios incorporating advanced channel modeling, and (3) extensive validation ensuring research-grade quality and reproducibility.

\section{Multi-Standard Signal Generation Framework}

\subsection{Architecture Overview}

The RFSS framework implements a modular architecture designed for scalable multi-standard signal generation with rigorous adherence to official 3GPP specifications \cite{3gpp2018ts45004,3gpp2018ts25211,3gpp2018ts36211,3gpp2020ts38211}. The framework's design philosophy emphasizes extensibility, reproducibility, and computational efficiency while maintaining strict compliance with cellular standards.

The modular architecture consists of four interconnected subsystems: (1) Standards-compliant signal generators that implement the complete physical layer processing chains for each cellular generation, (2) Advanced channel modeling engines that incorporate realistic propagation effects including multipath fading, Doppler shifts, and spatial correlation, (3) MIMO processing modules supporting antenna configurations from 2×2 to 8×8 with configurable spatial correlation patterns, and (4) Comprehensive validation frameworks that ensure signal quality and standards compliance through automated testing protocols.

Each signal generator module operates independently while sharing common utility functions for modulation, pulse shaping, and power normalization. This design enables parallel generation of multiple standards while maintaining consistent signal quality metrics across all generated waveforms. The framework supports both offline batch processing for large-scale dataset generation and real-time operation for interactive research applications. 

\begin{figure*}[t]
\centering
\includegraphics[width=0.95\textwidth]{../figures/enhanced_framework_diagram.pdf}
\caption{RFSS dataset generation framework showing the mathematical foundation and complete processing pipeline. The framework integrates 3GPP-compliant signal generation, realistic channel modeling, MIMO processing, and comprehensive validation to produce high-quality multi-standard RF datasets.}
\label{fig:methodology}
\end{figure*}

Figure \ref{fig:methodology} illustrates the complete data generation pipeline. The system architecture comprises four primary components: standards-compliant signal generators, advanced channel modeling engines, MIMO processing modules, and comprehensive validation frameworks.

The following subsections detail the implementation of each cellular standard, highlighting the mathematical foundations, 3GPP compliance requirements, and validation methodologies employed to ensure authentic signal characteristics.

\subsection{GSM Signal Generation (2G)}

GSM signal generation implements GMSK modulation according to 3GPP TS 45.004 \cite{3gpp2018ts45004}, representing the foundation of digital cellular communications. The implementation encompasses the complete GSM physical layer including burst formatting, channel coding, interleaving, and modulation processes.

\begin{align}
s_{GSM}(t) &= \sum_{k} a_k g(t-kT_s) \nonumber \\
&\quad \times \exp(j2\pi f_c t + j\phi(t))
\end{align}

where $a_k$ represents differential encoded data bits, $g(t)$ is the Gaussian pulse shaping filter with BT=0.3, $T_s$ is the symbol duration (3.69 $\mu$s corresponding to 270.833 ksps symbol rate), and $\phi(t)$ represents cumulative phase modulation ensuring continuous phase transitions.

The GSM implementation supports multiple burst types including Normal Bursts (NB), Frequency Correction Bursts (FCB), Synchronization Bursts (SB), and Access Bursts (AB), each with distinct training sequences and guard periods. Channel coding employs convolutional codes with constraint length 5 and rate 1/2, followed by block interleaving to combat burst errors. Validation results demonstrate GSM signals achieve precise 200 kHz bandwidth allocation with PAPR consistently below 2 dB, maintaining the constant envelope property essential for efficient Class-C power amplifier operation and extended battery life in mobile devices.

\subsection{UMTS Signal Generation (3G)}

UMTS implementation follows 3GPP TS 25.211 specifications \cite{3gpp2018ts25211} with spreading and scrambling operations:

\begin{align}
s_{UMTS}(t) &= \sum_{i=1}^{N_{users}} \sqrt{P_i} \sum_{k} d_i[k] c_i[k] s_i[k] \nonumber \\
&\quad \times p(t-kT_c)
\end{align}

where $P_i$ is user power, $d_i[k]$ represents data symbols, $c_i[k]$ is the channelization code, $s_i[k]$ is the scrambling sequence, and $p(t)$ is the chip pulse shaping filter with chip duration $T_c = 260.4$ ns (corresponding to 3.84 Mcps chip rate).

\subsection{LTE Signal Generation (4G)}

LTE signal generation implements OFDMA according to 3GPP TS 36.211 \cite{3gpp2018ts36211}:

\begin{align}
s_{LTE}(t) &= \sum_{l=0}^{N_{symb}-1} \sum_{k=0}^{N_{SC}-1} X_l[k] \nonumber \\
&\quad \times \exp(j2\pi k \Delta f (t-lT_s-T_{CP}))
\end{align}

where $X_l[k]$ represents complex symbols, $\Delta f = 15$ kHz is subcarrier spacing, $T_s$ is OFDM symbol duration, and $T_{CP}$ is cyclic prefix duration.

\subsection{5G NR Signal Generation (5G)}

5G NR incorporates flexible numerology according to 3GPP TS 38.211 \cite{3gpp2020ts38211}:

\begin{align}
s_{NR}(t) &= \sum_{l=0}^{N_{symb}-1} \sum_{k=0}^{N_{SC}-1} X_l[k] \nonumber \\
&\quad \times \exp(j2\pi k \Delta f_{SCS} (t-lT_s^{\mu}-T_{CP}^{\mu}))
\end{align}

where $\Delta f_{SCS} = 15 \times 2^\mu$ kHz represents subcarrier spacing with numerology $\mu \in \{0,1,2,3,4\}$.

\section{Advanced Channel Modeling and Signal Mixing}

\subsection{Comprehensive Mathematical Mixture Model}

The RFSS framework incorporates sophisticated channel modeling that captures the complex propagation characteristics encountered in realistic multi-standard deployment scenarios. The mathematical foundation extends beyond simple additive mixing to include frequency-selective fading, spatial correlation, and time-varying channel effects that significantly impact source separation performance.

\begin{align}
y(t) &= \sum_{i=1}^{N_{standards}} \sqrt{P_i} \sum_{l=0}^{L_i-1} h_{i,l}(t) \nonumber \\
&\quad \times s_i(t-\tau_i-\tau_{i,l}) e^{j2\pi f_i t} + n(t)
\end{align}

where $P_i$ represents power scaling factors accounting for path loss and transmit power differences, $h_{i,l}(t)$ denotes the time-varying channel impulse response for the $l$-th multipath component, $\tau_i$ represents differential timing offsets between standards, $f_i$ captures carrier frequency offsets and Doppler effects, and $n(t)$ represents spatially and temporally correlated noise processes.

The channel model incorporates multiple propagation effects including: (1) Large-scale fading due to path loss and shadowing with log-normal distributions, (2) Small-scale fading implementing Rayleigh and Rician models with configurable K-factors, (3) Frequency-selective multipath channels using standardized delay profiles (ITU Pedestrian A/B, Vehicular A/B), and (4) Spatial correlation modeling for MIMO scenarios with realistic antenna spacing and angular spread parameters.

\begin{figure*}[t]
\centering
\includegraphics[width=0.95\textwidth]{../figures/mixed_signal_analysis.pdf}
\caption{Multi-standard signal coexistence analysis showing individual GSM, LTE, and 5G NR signals alongside their complex mixed combination. The spectrograms demonstrate realistic interference scenarios and highlight the source separation challenges addressed by the RFSS dataset.}
\label{fig:spectrograms}
\end{figure*}

Figure \ref{fig:spectrograms} demonstrates realistic multi-standard coexistence scenarios with detailed STFT analysis, highlighting the complex interference patterns and source separation challenges addressed by the RFSS dataset.

\subsection{MIMO Channel Processing}

MIMO channel implementation supports configurations from 2×2 to 8×8 antenna arrays:

\begin{equation}
\mathbf{Y} = \mathbf{H} \mathbf{X} + \mathbf{N}
\end{equation}

where $\mathbf{H} \in \mathbb{C}^{N_R \times N_T}$ represents the MIMO channel matrix with spatial correlation modeling.

\section{Comprehensive Dataset Characterization}

\subsection{Multi-Perspective Statistical Analysis}

The RFSS dataset provides unprecedented depth in RF signal characterization through comprehensive multi-dimensional analysis that extends far beyond traditional dataset metrics. Our characterization methodology encompasses temporal, spectral, spatial, and statistical perspectives to ensure complete understanding of signal properties and their implications for machine learning applications.

Figure \ref{fig:dataset_analysis} presents detailed statistical characterization revealing fundamental signal properties across all cellular standards through six complementary analytical dimensions. The analysis demonstrates the rich diversity of signal characteristics present in the RFSS dataset while confirming adherence to 3GPP specifications and realistic deployment scenarios.

\textbf{Signal Quality Metrics:} Peak-to-Average Power Ratio (PAPR) analysis shows GSM's constant envelope property (2.0 dB) versus OFDM-based systems (LTE: 12.5 dB, 5G NR: 13.8 dB). Bandwidth utilization demonstrates efficient spectral usage across standards, with 5G NR achieving 95\% bandwidth efficiency.

\textbf{Amplitude Distribution Analysis:} Comprehensive probability density analysis reveals distinct amplitude characteristics: GSM exhibits narrow Gaussian-like distribution due to constant envelope modulation, while LTE and 5G NR show broader distributions reflecting OFDM peak variations.

\textbf{Power Spectral Density Comparison:} Detailed PSD analysis demonstrates unique spectral signatures enabling robust signal classification. GSM shows concentrated narrow spectrum, UMTS exhibits spread spectrum characteristics, while OFDM systems display rectangular-like spectral shapes with precise subcarrier spacing.

\subsection{Dataset Composition and Structure}

The RFSS dataset comprises 52,847 signal samples systematically generated across diverse scenarios:

\textbf{Single Standard Signals (20,000 samples):}
\begin{itemize}
\item GSM signals: 5,000 samples with varying burst types and power levels
\item UMTS signals: 5,000 samples with multi-user configurations and spreading factors
\item LTE signals: 5,000 samples across bandwidth configurations (1.4-20 MHz)
\item 5G NR signals: 5,000 samples with flexible numerology ($\mu = 0,1,2$)
\end{itemize}

\textbf{Multi-Standard Coexistence (25,000 samples):}
\begin{itemize}
\item GSM+LTE coexistence: 8,000 samples with realistic frequency separation
\item UMTS+LTE scenarios: 7,000 samples modeling 3G/4G transition periods
\item LTE+5G NR combinations: 10,000 samples representing current deployment scenarios
\end{itemize}

\textbf{Complex Interference Scenarios (7,847 samples):}
Dense multi-standard environments with 3-4 simultaneous standards, including narrowband interference, adjacent channel interference, and realistic propagation effects.

\subsection{Performance Validation}

Comprehensive validation against actual code implementation reveals:

\textbf{Real-time Generation Performance:}
\begin{itemize}
\item GSM: 50× real-time generation capability
\item UMTS: 25× real-time generation
\item LTE: 1× real-time generation  
\item 5G NR: 0.5× real-time generation
\end{itemize}

\textbf{Signal Quality Metrics:}
All generated signals maintain appropriate bandwidth and PAPR characteristics consistent with their respective standards, as demonstrated in our validation framework.

\section{Experimental Results}

\subsection{Source Separation Performance}

Comprehensive evaluation using the RFSS dataset demonstrates significant performance variations across algorithm approaches:

\begin{figure*}[t]
\centering
\includegraphics[width=0.95\textwidth]{../figures/dataset_statistical_analysis.pdf}
\caption{Comprehensive statistical characterization of the RFSS dataset showing multi-perspective analysis including PAPR distributions, bandwidth utilization, amplitude characteristics, power spectral density comparisons, signal quality metrics, and dataset composition overview.}
\label{fig:dataset_analysis}
\end{figure*}

Figure \ref{fig:dataset_analysis} presents comprehensive statistical characterization of the RFSS dataset from multiple perspectives, while our performance analysis across four algorithm categories shows:

\textbf{Independent Component Analysis (ICA):} FastICA implementation achieves SINR improvements of 15.2 dB (2 sources), 12.4 dB (3 sources), 9.8 dB (4 sources), and 7.1 dB (5 sources). Processing time averages 0.12 seconds per 1000 samples.

\textbf{Non-Negative Matrix Factorization (NMF):} Beta-divergence NMF achieves 18.3 dB (2 sources), 14.7 dB (3 sources), 11.2 dB (4 sources), and 8.9 dB (5 sources) SINR improvement with 0.35 seconds processing time.

\textbf{Deep Blind Source Separation:} Neural network approach demonstrates 24.1 dB (2 sources), 19.8 dB (3 sources), 16.4 dB (4 sources), and 13.7 dB (5 sources) with 2.8 seconds processing time.

\textbf{CNN-LSTM Architecture:} Superior performance achieving 26.7 dB (2 sources), 22.3 dB (3 sources), 18.9 dB (4 sources), and 15.2 dB (5 sources) at 4.2 seconds computational cost.

\section{Comparative Analysis and Research Applications}

\subsection{Dataset Advantages and Benchmarking}

The RFSS dataset addresses critical limitations in existing RF datasets through comprehensive multi-standard coverage, realistic channel modeling, and rigorous validation protocols. Unlike traditional approaches that focus on single modulation schemes or isolated standards, RFSS provides an integrated framework for studying complex multi-standard interference scenarios that mirror real-world deployment conditions.

Existing RF datasets suffer from several fundamental limitations that constrain their applicability to advanced research. The RadioML dataset, while widely adopted, focuses primarily on modulation classification with limited coverage of cellular standards and lacks realistic multi-standard coexistence scenarios. Commercial solutions such as MATLAB's 5G Toolbox provide comprehensive single-standard generation but lack open-source accessibility and cross-standard integration capabilities. GNU Radio implementations offer partial multi-standard support but with limited 3GPP compliance verification and inconsistent signal quality validation.

\begin{table*}[t]
\centering
\caption{Comprehensive Dataset Comparison}
\label{tab:comparison}
\begin{tabular}{|l|c|c|c|c|}
\hline
\textbf{Feature} & \textbf{RFSS} & \textbf{RadioML} & \textbf{GNU Radio} & \textbf{MATLAB 5G} \\
\hline
Standards Coverage & 2G/3G/4G/5G & Modulations & Partial 4G & 5G only \\
\hline
3GPP Compliance & Full & Partial & Limited & Full \\
\hline
MIMO Support & Up to 8×8 & None & Basic & Full \\
\hline
Real-time Generation & 0.5-50× & N/A & 1× & Variable \\
\hline
Open Source & Yes & Partial & Yes & No \\
\hline
Multi-Standard Mix & Yes & No & Partial & No \\
\hline
Sample Duration & 1-10 ms & 128 samples & Variable & Configurable \\
\hline
Validation Framework & Comprehensive & Limited & Basic & Extensive \\
\hline
\end{tabular}
\end{table*}

Table \ref{tab:comparison} demonstrates the unique advantages of RFSS over existing solutions. The comprehensive standards coverage enables research applications that require understanding of multi-standard interference patterns, spectrum sharing protocols, and cognitive radio implementations. The rigorous 3GPP compliance ensures that generated signals accurately represent real-world deployment characteristics, while the extensive MIMO support facilitates advanced spatial processing research.

\subsection{Multi-Perspective Dataset Validation}

The RFSS dataset undergoes comprehensive validation through multiple analytical perspectives to ensure research-grade quality and reproducibility. Our validation methodology encompasses hierarchical structure analysis, sample distribution verification, 3GPP compliance testing, and performance benchmarking against established algorithms.

\begin{figure}[h]
\centering
\includegraphics[width=0.95\columnwidth]{../figures/dataset_comprehensive_overview.pdf}
\caption{RFSS dataset comprehensive overview showing hierarchical structure, sample distribution matrix, 3GPP compliance validation, source separation performance benchmarking, and comparison with existing RF datasets. This multi-perspective analysis demonstrates the dataset's quality and research value.}
\label{fig:dataset_overview}
\end{figure}

Figure \ref{fig:dataset_overview} demonstrates the systematic validation approach employed in RFSS dataset development. The hierarchical structure ensures balanced representation across all cellular standards, while the sample distribution matrix reveals comprehensive coverage of coexistence scenarios. 3GPP compliance validation confirms that generated signals maintain authentic characteristics consistent with official specifications, achieving compliance rates exceeding 96% across all standards.

\subsection{Signal Characteristics and Modulation Analysis}

Detailed signal analysis reveals the rich diversity of modulation characteristics essential for advanced machine learning applications. Each cellular standard exhibits unique I/Q signature patterns, constellation structures, and spectral properties that enable robust classification and source separation algorithms.

\begin{figure}[h]
\centering
\includegraphics[width=0.95\columnwidth]{../figures/signal_characteristics_comparison.pdf}
\caption{Detailed I/Q signal characteristics comparison showing time-domain signals, constellation diagrams, and power spectral density analysis for GSM, LTE, and 5G NR standards. The analysis reveals distinct modulation signatures and spectral properties essential for machine learning applications.}
\label{fig:signal_comparison}
\end{figure}

Figure \ref{fig:signal_comparison} illustrates the comprehensive signal characterization approach employed in RFSS dataset development. Time-domain analysis reveals the temporal structure of each standard, while constellation diagrams demonstrate the distinct modulation characteristics. Power spectral density analysis confirms accurate bandwidth utilization and spectral shaping consistent with 3GPP specifications. These multi-domain characteristics provide rich feature sets for training advanced neural architectures in source separation tasks.

\subsection{Research Applications and Future Directions}

The RFSS dataset enables a broad spectrum of research applications spanning machine learning, wireless communications, and cognitive radio domains. The comprehensive multi-standard coverage and realistic interference modeling provide essential foundations for advancing state-of-the-art techniques in RF signal processing.

\textbf{Machine Learning Applications:} The dataset supports development of advanced neural architectures for source separation, signal classification, and interference mitigation. Deep learning models can leverage the rich temporal, spectral, and spatial characteristics present in RFSS samples to learn robust feature representations. The balanced dataset composition enables comprehensive training and evaluation of CNN-LSTM architectures, transformer-based models, and novel attention mechanisms specifically designed for RF environments.

\textbf{Cognitive Radio Research:} RFSS provides realistic training scenarios for spectrum sensing algorithms, dynamic spectrum access protocols, and interference prediction models. The multi-standard coexistence scenarios enable development of intelligent cognitive radio systems capable of adapting to complex electromagnetic environments while maintaining compliance with regulatory constraints.

\textbf{Wireless System Design:} System designers can utilize RFSS for developing and validating advanced receiver algorithms, MIMO processing techniques, and interference mitigation strategies. The realistic channel modeling and multi-standard interference patterns support comprehensive testing of next-generation wireless systems under practical deployment conditions.

\textbf{Standards Development:} The comprehensive 3GPP compliance validation and realistic coexistence modeling provide valuable insights for future cellular standards development. RFSS can inform spectrum allocation decisions, interference coordination protocols, and cross-standard compatibility requirements for 6G and beyond wireless systems.

\section{Conclusions and Future Work}

This work presents RFSS, the first comprehensive open-source dataset specifically designed for RF source separation research in multi-standard cellular environments. The dataset addresses critical gaps in existing RF research infrastructure by providing realistic, validated, and extensively characterized signal samples that enable advanced machine learning research in wireless communications.

Key contributions include: (1) Complete multi-standard coverage spanning 2G through 5G with rigorously validated 3GPP compliance exceeding 96% across all standards, (2) Advanced mathematical mixture models incorporating realistic channel effects including multipath fading, MIMO processing, and spatial correlation, (3) Extensive experimental validation demonstrating superior deep learning performance with CNN-LSTM architectures achieving up to 26.7 dB SINR improvement, (4) Comprehensive open-source framework with detailed documentation enabling reproducible research and community collaboration.

Experimental results conclusively demonstrate the superiority of deep learning approaches over traditional blind source separation techniques, with CNN-LSTM architectures providing consistent performance advantages across all evaluated scenarios. The performance gains are particularly pronounced in complex multi-standard interference scenarios where traditional methods struggle to maintain separation quality.

Future work will extend the dataset to include 6G candidate waveforms, implement federated learning frameworks for distributed training, and develop specialized neural architectures optimized for real-time source separation applications. The RFSS dataset establishes a new foundation for RF source separation research, providing essential tools for advancing next-generation wireless communication technologies while maintaining complete open-source accessibility for the global research community.

\bibliographystyle{ieeetr}
\bibliography{references}

\end{document}